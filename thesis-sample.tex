\documentclass[12pt,oneside]{report}

\usepackage{cnu-thesis}

% Common packages
\usepackage{amsmath, amssymb, amsfonts, mathtools, amsthm, bm}
\usepackage{physics, siunitx, mhchem, circuitikz, tikz}
\usepackage{listings, xcolor}
\lstset{
    basicstyle=\ttfamily\small,
    numbers=left,
    numberstyle=\tiny,
    breaklines=true,
    keywordstyle=\color{blue},
    commentstyle=\color{gray},
    stringstyle=\color{red}
}
\usepackage{graphicx, subcaption, booktabs}
\usepackage[backend=biber, style=authoryear, sorting=nyt]{biblatex}
\addbibresource{references.bib}
\usepackage{hyperref}

\begin{document}

\apcsabstract
{A STUDY OF SIMULATED WIND TUNNEL DATA}
{JANE DOE}
{2025}
{DR. ERIN HARPER, PH.D., ASSOCIATE PROFESSOR, DEPARTMENT OF PHYSICS, COMPUTER SCIENCE AND ENGINEERING}
{This thesis presents a computational and experimental study of airflow in a simulated wind tunnel environment. Using modern numerical techniques and small-scale experiments, results are compared and validated against theoretical models. Implications for applied physics and computational engineering are discussed.}

\thesistitlepage
{A STUDY OF SIMULATED WIND TUNNEL DATA}
{JANE DOE}
{2025}
{PROF. SAM SMITH}
{DR. LINDA JONES}
{DR. ROBERT TAYLOR}

\frontmatter
\dedication{To my family and mentors, for their unwavering support.}
\acknowledgements{I thank my advisor, my committee, and my peers for their invaluable guidance and encouragement.}

\tableofcontents
\listoftables
\listoffigures

\mainmatter

\chapter{Introduction}
Aerodynamics has long been a central theme in applied physics and engineering. This thesis explores a simulation of airflow using both numerical and experimental techniques.

According to Einstein's famous relation, $E = mc^2$ \cite{einstein1905}, energy and mass are equivalent. In computational engineering, such relations guide scaling and modeling choices.

\chapter{Methods}
A wind tunnel simulation was created using numerical integration and computational fluid dynamics software. The following equation was central to our model:
\begin{equation}
\nabla \cdot \vec{E} = \frac{\rho}{\epsilon_0}
\end{equation}

A sample circuit diagram is shown in Figure~\ref{fig:circuit}.

\cnuFigure{Circuit diagram of the measurement system.\label{fig:circuit}}{
\begin{circuitikz}
\draw (0,0) to[battery] (0,2)
      to[R=1\,\si{\kilo\ohm}] (2,2)
      to[short] (2,0)
      to[short] (0,0);
\end{circuitikz}
}

\chapter{Results}
The experiments produced the following summary (Table~\ref{tab:results}).

\cnuTable{Summary of airflow measurements.\label{tab:results}}{
\begin{tabular}{@{}lcc@{}}
\toprule
Condition & Mean Velocity (m/s) & Std Dev \\\\
\midrule
A & 10.2 & 0.5 \\\\
B & 12.4 & 0.8 \\\\
C & 9.8  & 0.6 \\\\
\bottomrule
\end{tabular}
}

\chapter{Conclusions}
This study demonstrates that simulated airflow results can be consistent with theoretical predictions when carefully calibrated. Future work will expand to larger tunnel scales.

\printbibliography[title={LITERATURE CITED}]
\blankpage

\appendix
\chapter{Additional Data}
Additional plots and data tables are included here.

\end{document}
